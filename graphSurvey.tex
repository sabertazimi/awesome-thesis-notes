% !TEX program = xelatex
% !BIB program = bibtex

\documentclass[UTF8,12pt,a4paper]{article}

\usepackage{ctex}

% layout
\usepackage[left=2.5cm,right=2.5cm]{geometry}
\usepackage{paralist}     % for compactitem environment
\usepackage{indentfirst}  % ident the first paragraph
\linespread{1.25}
% \makeatletter
% \def\@seccntformat#1{%
%   \expandafter\ifx\csname c@#1\endcsname\c@section
%   Section \thesection:
%   \else
%   \csname the#1\endcsname\quad
%   \fi}
% \makeatother
 
% page headings
\usepackage{fancyhdr}
\setlength{\headheight}{15.2pt}
\pagestyle{fancy}
\lhead{\leftmark}
\rhead{M201873026 Yilong Liu}
\cfoot{\thepage}
% \makeatletter
% \let\headauthor\@author
% \makeatother

% url/ref
\usepackage{hyperref}
\hypersetup{
  colorlinks,
  citecolor=black,
  filecolor=black,
  linkcolor=black,
  urlcolor=black,
  pdfauthor={Yilong Liu},
  pdftitle={Graph Processing: state-of-the-art and research challenges}
}

% vertical centering title page
\usepackage{titling}
\renewcommand\maketitlehooka{\null\mbox{}\vfill}
\renewcommand\maketitlehookd{\vfill\null}

% table of contents
\usepackage{tocloft}
\renewcommand\cftsecfont{\normalfont}
\renewcommand\cftsecpagefont{\normalfont}
\renewcommand{\cftsecleader}{\cftdotfill{\cftsecdotsep}}
\renewcommand\cftsecdotsep{\cftdot}
\renewcommand\cftsubsecdotsep{\cftdot}
\renewcommand\cftsubsubsecdotsep{\cftdot}
\renewcommand{\contentsname}{\hfill\bfseries\Large Contents\hfill}   
\setlength{\cftbeforesecskip}{10pt}

% figures
\usepackage{graphicx}
\graphicspath{figures/}
% \newcommand\figureht{\dimexpr
%   \textheight-3\baselineskip-\parskip-.2em-
%   \abovecaptionskip-\belowcaptionskip\relax}


% tables
\usepackage{caption} 
% \captionsetup[table]{skip=10pt}

% math, algorithms, code
\usepackage{amsmath,amssymb,url}
\usepackage{algorithm,algorithmicx,algpseudocode}
\usepackage{listings}

\lstset{
   extendedchars=true,
   basicstyle=\footnotesize\ttfamily,
   showstringspaces=false,
   showspaces=false,
   numbers=left,
   numberstyle=\footnotesize,
   numbersep=9pt,
   tabsize=2,
   breaklines=true,
   showtabs=false,
   captionpos=b
}

% bibliography
\usepackage[super,square,comma,sort]{natbib} % for \citet and \citep
\renewcommand{\refname}{References}
% \begin{filecontents}{report.bib}
% \end{filecontents} 

% appendix
\usepackage{appendix}

\title{Survey \\ \bigskip \textbf{Graph Processing: state-of-the-art and research challenges}}
\author{Huazhong University of Science and Technology\\ School of Computer Science and Technology\\ M1801\\ M201873026\\ Yilong Liu}
\date{\today}

\begin{document}

\pagenumbering{gobble} % no page number
\maketitle
\newpage
% \null\thispagestyle{empty}
% \newpage

% \pagenumbering{roman}
% \section*{Abstract}\sectionmark{Abstract}
% \addcontentsline{toc}{section}{Abstract}
% \addcontentsline{toc}{section}{\protect\numberline{}Abstract}
% \newpage
% \pagenumbering{gobble} % no page number

\pagenumbering{roman}
\tableofcontents
\newpage
% \null\thispagestyle{empty}
% \newpage

\pagenumbering{arabic}

\section{FPGA}

FPGA in Data Center~\cite{DBLP:conf/isca/PutnamCCCCDEFGGHHHHKLLPPSTXB14}.

\begin{compactitem}
  \item cpu 通用
  \item gpu 高吞吐量
  \item fpga 低延迟
\end{compactitem}

FPGA 比 CPU 和 GPU 能效高,体系结构上的根本优势是无指令、无需共享内存.
我们发现通过 OpenCL 写 DRAM、启动 kernel、读 DRAM 一个来回,需要 1.8 毫秒。而通过 PCIe DMA 来通信,却只要 1~2 微秒.
在数据中心里 FPGA 的主要优势是稳定又极低的延迟,适用于流式的计算密集型任务和通信密集型任务.
FPGA 定义为通信的「大管家」,不管是服务器跟服务器之间的通信,虚拟机跟虚拟机之间的通信,进程跟进程之间的通信,CPU 跟存储设备之间的通信,都可以用 FPGA 来加速.
不管通信还是机器学习、加密解密,算法都是很复杂的,
如果试图用 FPGA 完全取代 CPU,势必会带来 FPGA 逻辑资源极大的浪费,也会提高 FPGA 程序的开发成本.
更实用的做法是 FPGA 和 CPU 协同工作,局部性和重复性强的归 FPGA,复杂的归 CPU.

\clearpage

\section{Graph Survey Template}

\begin{compactitem}
  \item data presentation
  \item data source
  \item data manipulation
  \item open source
  \item programming model
  \item partioning
  \item static and dynamic graphs
  \item batch and streaming
  \item algorithms
  \item community activity
  \item users
  \item books 
  \item distinction points
\end{compactitem}

\clearpage


% \begin{table}
%   \begin{small}    
%     \caption{This is for long caption.}
%     \label{tab:table}
%     \begin{center}
%       \begin{tabular}[c]{l|l}
%         \hline
%         \multicolumn{1}{c|}{\textbf{xx}} & 
%         \multicolumn{1}{c}{\textbf{xx}} \\
%         \hline
% 	      a & b \\
% 	      c & d \\
% 	      e & f \\
%         \hline
%       \end{tabular}
%     \end{center}
%   \end{small}
% \end{table}

% \begin{algorithm}
%   \floatname{algorithm}{算法}
% 	\algrenewcommand\algorithmicrequire{\textbf{输入:}}
% 	\algrenewcommand\algorithmicensure{\textbf{输出:}}
% 	\caption{xxxxxxxxxx}
% 	\label{alg:main}
%   \begin{algorithmic}[1]
%     \State \textbf{return} $state$
% 	\end{algorithmic}  
% \end{algorithm}

% \begin{equation}
%   \text{UCT} = \frac{w_i}{s_i} + c\sqrt{\frac{\ln{s_p}}{s_i}}
%   \label{eq:uct}
% \end{equation}

\bibliographystyle{unsrt}
\bibliography{bibs/graph}
\addcontentsline{toc}{section}{References}
\newpage

\end{document}

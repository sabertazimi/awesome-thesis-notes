% !TEX program = xelatex
% !BIB program = bibtex

\documentclass[UTF8,12pt,a4paper]{article}
\usepackage{ctex}

% layout
\usepackage[left=2.5cm,right=2.5cm]{geometry}
\usepackage{paralist}     % for compactitem environment
\usepackage{indentfirst}  % ident the first paragraph
\linespread{1.25}
% \makeatletter
% \def\@seccntformat#1{%
%   \expandafter\ifx\csname c@#1\endcsname\c@section
%   Section \thesection:
%   \else
%   \csname the#1\endcsname\quad
%   \fi}
% \makeatother
 
% page headings
\usepackage{fancyhdr}
\setlength{\headheight}{15.2pt}
\pagestyle{fancy}
\lhead{\leftmark}
\rhead{M201873026 Yilong Liu}
\cfoot{\thepage}
% \makeatletter
% \let\headauthor\@author
% \makeatother

% url/ref
\usepackage{hyperref}
\hypersetup{
  colorlinks,
  citecolor=black,
  filecolor=black,
  linkcolor=black,
  urlcolor=black,
  pdfauthor={Yilong Liu},
  pdftitle={Graph Processing: state-of-the-art and research challenges}
}

% vertical centering title page
\usepackage{titling}
\renewcommand\maketitlehooka{\null\mbox{}\vfill}
\renewcommand\maketitlehookd{\vfill\null}

% table of contents
\usepackage{tocloft}
\renewcommand\cftsecfont{\normalfont}
\renewcommand\cftsecpagefont{\normalfont}
\renewcommand{\cftsecleader}{\cftdotfill{\cftsecdotsep}}
\renewcommand\cftsecdotsep{\cftdot}
\renewcommand\cftsubsecdotsep{\cftdot}
\renewcommand\cftsubsubsecdotsep{\cftdot}
\renewcommand{\contentsname}{\hfill\bfseries\Large Contents\hfill}   
\setlength{\cftbeforesecskip}{10pt}

% figures
\usepackage{graphicx}
\graphicspath{figures/}
% \newcommand\figureht{\dimexpr
%   \textheight-3\baselineskip-\parskip-.2em-
%   \abovecaptionskip-\belowcaptionskip\relax}


% tables
\usepackage{caption} 
% \captionsetup[table]{skip=10pt}

% math, algorithms, code
\usepackage{amsmath,amssymb,url}
\usepackage{algorithm,algorithmicx,algpseudocode}
\usepackage{listings}

\lstset{
   extendedchars=true,
   basicstyle=\footnotesize\ttfamily,
   showstringspaces=false,
   showspaces=false,
   numbers=left,
   numberstyle=\footnotesize,
   numbersep=9pt,
   tabsize=2,
   breaklines=true,
   showtabs=false,
   captionpos=b
}

% bibliography
\usepackage[super,square,comma,sort]{natbib} % for \citet and \citep
\renewcommand{\refname}{References}
% \begin{filecontents}{report.bib}
% \end{filecontents} 

% appendix
\usepackage{appendix}

\title{Survey \\ \bigskip \textbf{Graph Processing: state-of-the-art and research challenges}}
\author{Huazhong University of Science and Technology\\ School of Computer Science and Technology\\ M1801\\ M201873026\\ Yilong Liu}
\date{\today}

\begin{document}

\pagenumbering{gobble} % no page number
\maketitle
\newpage
% \null\thispagestyle{empty}
% \newpage

% \pagenumbering{roman}
% \section*{Abstract}\sectionmark{Abstract}
% \addcontentsline{toc}{section}{Abstract}
% \addcontentsline{toc}{section}{\protect\numberline{}Abstract}
% \newpage
% \pagenumbering{gobble} % no page number

\pagenumbering{roman}
\tableofcontents
\newpage
% \null\thispagestyle{empty}
% \newpage

\pagenumbering{arabic}

\section{FPGA}

FPGA in Data Center~\cite{DBLP:conf/isca/PutnamCCCCDEFGGHHHHKLLPPSTXB14}.

\begin{compactitem}
  \item cpu 通用
  \item gpu 高吞吐量
  \item fpga 低延迟
\end{compactitem}

FPGA 比 CPU 和 GPU 能效高,体系结构上的根本优势是无指令、无需共享内存.
我们发现通过 OpenCL 写 DRAM、启动 kernel、读 DRAM 一个来回,需要 1.8 毫秒。而通过 PCIe DMA 来通信,却只要 1~2 微秒.
在数据中心里 FPGA 的主要优势是稳定又极低的延迟,适用于流式的计算密集型任务和通信密集型任务.
FPGA 定义为通信的「大管家」,不管是服务器跟服务器之间的通信,虚拟机跟虚拟机之间的通信,进程跟进程之间的通信,CPU 跟存储设备之间的通信,都可以用 FPGA 来加速.
不管通信还是机器学习、加密解密,算法都是很复杂的,
如果试图用 FPGA 完全取代 CPU,势必会带来 FPGA 逻辑资源极大的浪费,也会提高 FPGA 程序的开发成本.
更实用的做法是 FPGA 和 CPU 协同工作,局部性和重复性强的归 FPGA,复杂的归 CPU.
\clearpage

\section{Workload Characterization}

\begin{compactitem}
  \item Memory bandwidth is not fully utilized.
  there is the potential for significant performance improvement
  on graph codes with current off-chip memory systems.
  \item Graph codes exhibit substantial locality.
  Optimized graph codes experience a moderately
  high last-level cache (LLC) hit rate.
  \item Reorder buffer size limits achievable memory throughput.
  The relatively high LLC hit rate implies
  many instructions are executed for each LLC miss.
  These instructions fill the reorder buffer in the core,
  preventing future loads that will miss in the LLC from issuing early,
  resulting in unused memory bandwidth.
  \item Multithreading has limited potential for graph processing
  Likely achievable performance with only a modest number (2) of threads per core.
\end{compactitem}

Because message-passing is far less efficient than accessing memory in contemporary systems,
the efficiency of each core in a cluster is on average
one to two orders-of-magnitude lower than cores in shared-memory nodes.
This communication-bound behavior has led to
a single SSD Node is able to outperform a medium-sized cluster~\cite{DBLP:conf/osdi/KyrolaBG12}.

Graph algorithms have their scaling hampered by
load imbalance, synchronization overheads, and non-uniform memory access (NUMA) penalties.
Different input graph sizes and topologies can lead to
very different conclusions for algorithms and architectures
\clearpage

\section{DRAM}
SRAM 容量太难做大, CPU 一半以上的面积都用来做 SRAM.
SRAM 面积越大, 速度就越慢.
SRAM 做到 GB 容量几乎是不可能完成的任务.
\subsection{Basis of DRAM}
\subsubsection{Terminology}
\begin{compactitem}
  \item RAS: Row Address Strobe (ACT Activate DRAM page/row Command)
  \item CAS: Column Address Strobe (READ Command)
  \item DDR: Double-Data Rate transfers data
  on both rising and falling edge of the clock
  \item Channel: 一组数据信号线、对应几个槽位、对应几根内存条称为一个 channel
  \item Bank: read out all words in parallel, 一个 bank 包含 N 个 DRAM Array
  \item DIMM: Dual Inline Memory Module 双通道内存
  \item channel>DIMM>rank>chip>bank>row/column
\end{compactitem}
\subsubsection{Address Mapping Scheme}
如表~\ref{tab:dram_address_mapping}所示,
$\text{Memory Capacity} = K*L*B*R*C*V$,
$N = CV/Z$, $CV = NZ$.
采用 open page (keep page in DRAM row buffer) 时,
映射为 k:l:r:b:n:z (可扩展) /r:l:b:n:k:z (高并行) (n 也等于 column, z 也等于 offset).
采用 close page (immediately close page in DRAM row buffer) 时,
映射为 k:l:r:n:b:z/r:n:l:b:k:z.
Accessing different rows from one bank is slowest.
\begin{table}
  \begin{small}
    \caption{DRAM Address Mapping Parameters}
    \label{tab:dram_address_mapping}
    \begin{center}
      \begin{tabular}[c]{l|l}
        \hline
        \multicolumn{1}{c|}{\textbf{Symbol}} & 
        \multicolumn{1}{c}{\textbf{Description}} \\
        \hline
        K & \# of channels in system \\
        L & \# of ranks per channel \\
        B & \# of banks per rank \\
        R & \# of rows per bank \\
        C & \# of columns per row \\
        V & \# of bytes per column \\
        Z & \# of bytes per cache line \\
        N & \# of cache lines per row \\
        \hline
      \end{tabular}
    \end{center}
  \end{small}
\end{table}
\subsubsection{Delay Time}
一条访存指令发到内存控制器, 它的访存延时是存在不同的可能性:
\begin{compactitem}
  \item row buffer hit: 从 row buffer 到把数据放在数据总线上的时延,大约 20 ns
  \item empty row buffer: 从电容到 sense amplifier 再到 row buffer 的时序 + 从 row buffer 到数据总线时间,大约40ns
  \item row buffer conflict: 写回时延 + empty row buffer delay, 大约60ns
  \item CL: CAS Latency, 从 CAS 与读取命令发出到第一笔数据输出的这段时间 (READ -> data)
  \item tRCD: RAS 到 CAS 时延 (Active -> READ)
  \item 要切换另一行,要发 precharge 命令 (close page/row), 把数据写到 cell 里去.
  关闭一个行需要时间, 这个时间称为 tRP, 发送 PRE/PREA 命令后 tRP 时间才可以发 ACT 命令.
\end{compactitem}
\subsection{Memory Controller}
来自 CPU 的请求以执行顺序缓冲进入 transaction queue,
这些请求被转换为 DRAM 命令并放入 command queue.
\subsection{Landscape of DRAM-based memory}
\begin{table}
  \begin{small}
    \caption{Landspace of DRAM-based memory~\cite{DBLP:journals/cal/KimYM16}}
    \label{tab:memory_landscape}
    \begin{center}
      \begin{tabular}[c]{l|l}
        \hline
        \multicolumn{1}{c|}{\textbf{Segment}} & 
        \multicolumn{1}{c}{\textbf{DRAM Standards \& Architectures}} \\
        \hline
        Commodity & DDR3;DDR4 \\
        Low-Power & LPDDR3;LPDDR4 \\
        Graphics & GDDR5 \\
        Performance & eDRAM;RLDRAM3;WIO;WIO2;MCDRAM \\
        3D-Stakced & HBM;HMC;SBA/SSA;Staged Reads;RAIDR \\
        Academic & SALP/SARP;AL/TL-DRAM;RowClone;Half-DRAM;Row-Buffer Decoupling \\
        \hline
      \end{tabular}
    \end{center}
  \end{small}
\end{table}
\clearpage

\section{Graph Survey Template}
\begin{compactitem}
  \item data presentation
  \item data source
  \item data manipulation
  \item open source
  \item programming model
  \item partioning
  \item static and dynamic graphs
  \item batch and streaming
  \item algorithms
  \item community activity
  \item users
  \item books 
  \item distinction points
\end{compactitem}
\clearpage


\bibliographystyle{unsrt}
\bibliography{bibs/graph}
\addcontentsline{toc}{section}{References}
\newpage

\end{document}
